% !TEX root = ../main.tex
\chapter{Introduction}
%max 2 pages

The LowLevelAnalyzer (LoLA) is a model checking tool for petri nets, but as of today it utilizes only a single thread in productive use.

\textbf{A Petri Net excurs}:
A petri net is a mathematical construction to describe the behavior of distributed systems. Petri nets where first defined by Carl Adam Petri\cite{petri1962kommunikation}. A Petri net is defined by a five tupel: (P,T,F,W,m0). It consist of \textbf{places} (P) that can hold infinit \textbf{tokens} (or marks). The initial amount of tokens on each place is defined by the initial \textbf{marking} (m0). Places can be connected with transitions (T) via directed \textbf{edges} (F for flow). Places (transitions) can not be connected to other places (transitions). 

If all edges that point to a transition are connected to a place that hold an equal or greater amount of tokens than the connecting edges \textbf{weight} (W), this transition can \textbf{fire} at an arbitrary time point. Fireing a transition will \textbf{consume} tokens from all places that point \textbf{to} the transition with a connecting edge, and it will \textbf{produce} tokens on all places where an edge points \textbf{from} this transition. The produced and consumed amount of tokens is equal to the corresponding edge weight.

With the complete mathematical background, Petri Nets can be used to check a variety of their properties. Figure \ref{petrNetExamples} shows some examples of Petri Nets.

\tikzset{
  place/.style={
    circle,
    thick,
    draw=blue!75,
    fill=blue!20,
    minimum size=6mm
  },
  transition/.style={
    rectangle,
    thick,
    draw=black,
    %fill=black,
    minimum width=8mm,
    minimum height=4.5mm,
    inner ysep=2pt
    }
  }            

\begin{tikzpicture}[node distance=1.3cm,>=stealth',bend angle=45,auto]

    \begin{scope}
        \node [place,tokens=1] (w1)                                    {};
        \node [place] (c1) [below of=w1]                      {};
        \node [place] (s)  [below of=c1,label=above:$s\le 3$] {};
        \node [place] (c2) [below of=s]                       {};
        \node [place,tokens=1] (w2) [below of=c2]                      {};

        \node [transition] (e1) [left of=c1] {}
        edge [pre,bend left]                  (w1)
        edge [post,bend right]                (s)
        edge [post]                           (c1);

        \node [transition] (e2) [left of=c2] {}
        edge [pre,bend right]                 (w2)
        edge [post,bend left]                 (s)
        edge [post]                           (c2);

        \node [transition] (l1) [right of=c1] {}
        edge [pre]                            (c1)
        edge [pre,bend left]                  (s)
        edge [post,bend right] node[swap] {2} (w1);

        \node [transition] (l2) [right of=c2] {}
        edge [pre]                            (c2)
        edge [pre,bend right]                 (s)
        edge [post,bend left]  node {2}       (w2);
    \end{scope}

    \begin{scope}[xshift=6cm]
        \node [place,tokens=1] (w1)                                    {};
        \node [place] (c1) [below of=w1]                      {};
        \node [place] (s)  [below of=c1,label=above:$s\le 3$] {};
        \node [place] (c2) [below of=s]                       {};
        \node [place,tokens=1] (w2) [below of=c2]                      {};

        \node [transition] (e1) [left of=c1] {}
        edge [pre,bend left]                  (w1)
        edge [post,bend right]                (s)
        edge [post]                           (c1);

        \node [transition] (e2) [left of=c2] {}
        edge [pre,bend right]                 (w2)
        edge [post,bend left]                 (s)
        edge [post]                           (c2);

        \node [transition] (l1) [right of=c1] {}
        edge [pre]                            (c1)
        edge [pre,bend left]                  (s)
        edge [post,bend right] node[swap] {2} (w1);

        \node [transition] (l2) [right of=c2] {}
        edge [pre]                            (c2)
        edge [pre,bend right]                 (s)
        edge [post,bend left]  node {2}       (w2);
    \end{scope}
\end{tikzpicture}

\textbf{LoLA}:
LoLAs development startet in 1998. It was aimed to be used by third party tools to check properties of petri nets\cite{schmidt2000lola}. Since then it was steadily updated to compete with other state of the art tools. Recurring prizes in a model checking contest with focus on petri nets suggest a success in this attempt\cite{MCC2017}.
The internal property evaluation however, is still most performant single threaded. To evaluate a property of a petri net, LoLA searches all necessary net states that can be reached from the initial one. The search is a depth first search on a directed graph, which is discovered during the search itself.

For the specialized search in LoLA, a previous attempt of parallelization exists. Unfortunately the performance is usually worse then the single threaded approach. It can even get worse the more threads are used.

In this work we will explore LoLA to find the part which is not performing as expected. The most important task, is to find the bottle neck. We will measure the performance of LoLA in different ways, so that we can compare the base version with an improved one that we will develop. The improved version should hopefully outperform the single threaded algorithm with a reasonable amount of threads. Additionally a desirable result would show a linear scaling of the performance with the amount of threads.

%TODO: explain the contents

% A thesis statement should do the following:

% Explain the readers how you interpret the subject of the research
% Tell the readers what to expect from your paper
% Answer the question you were asked
% Present your claim which other people may want to dispute