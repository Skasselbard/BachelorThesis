
\documentclass[10pt]{beamer} % die 10pt sollten festgelegt bleiben, da dies die Groesse der Mathematikschrift etc. beeinflusst

\usepackage[ngerman]{babel}  % deutsche Bezeichnungen und Trennung etc
\usepackage{hyperref, tikz, pgfplots}        % interne Hyperlinks
\usepackage[utf8]{inputenc}

\usepackage[ief,footuni, headframelogo]{./unirostock/beamerthemeRostock}

%TODO:Deckblattanordnung überarbeiten
\title{\\\\\\\\Parallele Zustandsraumsuche}
\subtitle{Bachelor Verteidigung}
\author{\textsc{Tom Meyer}}
\date{27. März 2018}
\institute{Universität Rostock, Institut für Informatik}
%\titlegraphic{\begin{center}Platz f\"ur ein Logo\\und anderes\end{ce
\footinstitute{Fakultät für Informatik und Elektrotechnik, Institut für Informatik}
% eigenes Logo oben rechts hinzufuegen (bitte auf vernuenftiges Format achten - ein zu hohes Logo verschiebt das Layout)
%\renewcommand{\mylogo}{\includegraphics[width=18.5mm]{institutslogo}}

\begin{document}

\begin{frame}% Titelseite
  \titlepage
\end{frame}

\begin{frame}{Struktur der Vortrages}{Damit der H\"orer auch ein wenig durchsieht}
  \tableofcontents[pausesections]
\end{frame}

\begin{frame}{Motivation}
  %TODO: Bild als erste Folie
    \begin{enumerate}
      \item Was: LoLA
      \item Warum: parallel ist langsam
      \item evtl peek aufs ergebnis
    \end{enumerate}
\end{frame}

\begin{frame}{Background}
  \begin{itemize}
    \item Petri Netze
    \item Wie sah LoLA vorher aus
    \item DFS in LoLa
    \item Benchmarking
  \end{itemize}
\end{frame}

\begin{frame}{Compare And Swap}
  \begin{itemize}
    \item 
  \end{itemize}
\end{frame}

\begin{frame}{CAS Auswertung}
  %%
\end{frame}

\begin{frame}{Profiling Einführung}
  %%
\end{frame}

\begin{frame}{Profiling Ergebnisse}
  %%
\end{frame}

\begin{frame}{Mara}
  %%
\end{frame}

\begin{frame}{Profiling nach Mara Integration}
  %%
\end{frame}

\begin{frame}{Zusammenfassung}
  
\end{frame}

\begin{frame}{Auswertung}
  
\end{frame}


%%%%%%%%%%%%%%%%%%%%%%%%%%%%%%%%%%%%%%%%%%%%%%%%%%%%%%%%%%%%%%%%%%%%%%%
%%%%%%%%%%%%%%%%%%%%%%%%%%%%%%%%%%%%%%%%%%%%%%%%%%%%%%%%%%%%%%%%%%%%%%%
\section{Einige Beispielfolien}


%%%%%%%%%%%% Beispielfolie aus dem BeamerUsersguide %%%%%%%%%%%%%%%%%%%
%%%%%%%%%%%%%%%%%%%%%%%%%%%%%%%%%%%%%%%%%%%%%%%%%%%%%%%%%%%%%%%%%%%%%%%
\subsection{Beispiel aus beamerusersguide.pdf}
\begin{frame}
  \frametitle{There Is No Largest Prime Number}
  \framesubtitle{The proof uses \textit{reductio ad absurdum}.}
  \begin{theorem}
    There is no largest prime number.
  \end{theorem}
  \begin{proof}
    \begin{enumerate}
    \item<1-| alert@1> Suppose $p$ were the largest prime number.
    \item<2-> Let $q$ be the product of the first $p$ numbers.
    \item<3-> Then $q+1$ is not divisible by any of them.
    \item<1-> Thus $q+1$ is also prime and greater than $p$.\qedhere
    \end{enumerate}
  \end{proof}
\end{frame}



\end{document}

