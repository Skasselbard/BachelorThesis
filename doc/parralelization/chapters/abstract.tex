% !TEX root = ../main.tex
\section*{Parallele Zustandsraumsuche}
Mit dem LowLevelAnalyzer (LoLA) ist es durch Auswertung des Zustandsraums möglich, Eigenschaften von Petri Netzen auszuwerten. Die dafür notwendige Zustandsraumsuche wurde dabei bisher durch einen sequentiellen Algorithmus vorgenommen.

In den letzten Jahren haben sich im Hardwarebereich jedoch mehrkern Prozessorarchitekturen durchgesetzt. LoLAs sequentieller Ansatz kann deshalb oft das Rechenpotential moderner Maschinen nicht mehr komplett ausnutzen. Auch ein Versuch den vorhandenen Algorithmus zu parallelisieren konnte bisher noch keine Performanceverbesserung hervorrufen.

In dieser Arbeit wird gezeigt das es möglich ist, eine Performanceverbesserung durch Nutzung von mehreren Threads, zu erzielen. Wir werden den vorangegangenen Versuch analysieren und die Implementation so anpassen, dass eine Skalierung der Performance mit der Anzahl der genutzten Threads zu beobachten ist. Die Verbesserung wird dabei an einem verbreiteten Petri Netz gezeigt.

\section*{Parallel state space search}
The LowLevelAnalyzer (LoLA) is a useful tool to analyze properties of a Petri net by expanding and evaluating its state space. To search for new states LoLA makes use of a sequential depth first search algorithm.

However recent hardware development introduced multi-core architectures that cannot be fully exploited with LoLAs implementation. For this reason the previously single threaded search algorithm was adapted to use multiple threads. Unfortunately the sequential algorithm always beats the parallel in the matter of execution time for a yet unknown reason.

In this work it is shown that it is possible to utilize multiple threads for the state space expansion. For that we will analyze the previous parallel implementation of LoLA and improve it to achieve a performance scaling with the number of used threads. We will show that a commonly used Petri net will benefit from the new implementation.



\vfill

\begin{tabular}{ll}
	\bfseries Betreuer: & \parbox[t]{10cm}{\betreuer }\vspace{5mm} \\
	\bfseries Tag der Ausgabe: & 13.10.2017 \\
	\bfseries Tag der Abgabe: & 02.03.2018 \\
\end{tabular}