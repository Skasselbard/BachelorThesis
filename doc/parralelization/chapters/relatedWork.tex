\chapter{Related Work}

Compare and swap
\begin{itemize}
    \item atomic function
    \item implemented as atomic hardware instruction
\end{itemize}

Semaphore\cite{mullender2008semaphores}
\begin{itemize}
    \item non negative integer
    \item two operations
    \begin{itemize}
        \item V $\rightarrow$ increase (semaquire)
        \item P $\rightarrow$ decrease (semrelease)
    \end{itemize}
\end{itemize}

Copy-on-write-like techniques (including read-copy-update)
provide impressively high performance for read-only workloads
but very low performance for update workloads. \cite{gramoli2015more}


NC steht in der Informatik als Abkürzung für Nick's Class (nach Nick Pippenger), die Komplexitätsklasse der parallel effizient lösbaren Entscheidungsprobleme. https://de.wikipedia.org/wiki/NC_(Komplexit%C3%A4tsklasse)


Randomized NC
https://books.google.de/books?id=ThBR22HOjMcC&pg=PA41&lpg=PA41&dq=randomized+NC&source=bl&ots=jFwyJt80ub&sig=zb4TpnAULxKmN8t3SRl-CLDLWn4&hl=de&sa=X&ved=0ahUKEwil-6Wn2KnXAhURFuwKHXxlC1sQ6AEIPzAD#v=onepage&q&f=false

NC stands for \"Nick's Class\", after Nicholas Pippenger, who first identified it and suggested
that it contains precisely the problems we think of as having \"good\" parallel algorithms [45]. NC
remains the same across a wide variety of machine models, although the subclasses NCI" may vary
[14]. It is easy to see that NC P, but whether NC = Pis a famous open problem (see Section 6
below). \cite{freeman1991parallel}

The intuition is also similar: RNCis the set of
all problems that have fast parallel randomized algorithms. \cite{freeman1991parallel}

Profiling:
\begin{itemize}
    \item gprof braucht exit() statt _exit und liefert keine ergebnisse
    \item valgrind sequenziert die einzelnen threads
    \item operf (oprofile) funktioniert nicht out of the box -> permission errors unter root, kann nicht ohne weiteres auf ebro gebaut oder installiert werden
    \item perf scheint zu funktionieren
    \item profiling ist teilweise hardwareabhängig und benötigt möglicherweise debug kernel
\end{itemize}

perf install and sources
https://askubuntu.com/questions/50145/how-to-install-perf-monitoring-tool#50148

Erkenntnisse:
\begin{itemize}
    \item dfs ist schwer zu parallelisieren
    \item bis in 80er eher als unparallelisierbar gehalten
    \item für lexikalische graphen nicht parallelisierbar
    \item parallele Algorithmen für bestimmte graphklassen gefunden (planare, disjoint)
    \item in Nicks Class (NC)
    \item auch algorithmus für allgemeine gerichtete graphen
    \item menge der knoten muss dabei allerdings for suche bekannt sein und kann nicht während der suche aufgedeckt werden
\end{itemize}