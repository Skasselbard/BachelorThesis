Compare and swap
\begin{itemize}
    \item atomic function
    \item implemented as atomic hardware instruction
\end{itemize}

Semaphore\cite{mullender2008semaphores}
\begin{itemize}
    \item non negative integer
    \item two operations
    \begin{itemize}
        \item V $\rightarrow$ increase (semaquire)
        \item P $\rightarrow$ decrease (semrelease)
    \end{itemize}
\end{itemize}

Copy-on-write-like techniques (including read-copy-update)
provide impressively high performance for read-only workloads
but very low performance for update workloads. \cite{gramoli2015more}


NC steht in der Informatik als Abkürzung für Nick's Class (nach Nick Pippenger), die Komplexitätsklasse der parallel effizient lösbaren Entscheidungsprobleme. https://de.wikipedia.org/wiki/NC_(Komplexit%C3%A4tsklasse)


Randomized NC
https://books.google.de/books?id=ThBR22HOjMcC&pg=PA41&lpg=PA41&dq=randomized+NC&source=bl&ots=jFwyJt80ub&sig=zb4TpnAULxKmN8t3SRl-CLDLWn4&hl=de&sa=X&ved=0ahUKEwil-6Wn2KnXAhURFuwKHXxlC1sQ6AEIPzAD#v=onepage&q&f=false

NC stands for \"Nick's Class\", after Nicholas Pippenger, who first identified it and suggested
that it contains precisely the problems we think of as having \"good\" parallel algorithms [45]. NC
remains the same across a wide variety of machine models, although the subclasses NCI" may vary
[14]. It is easy to see that NC P, but whether NC = Pis a famous open problem (see Section 6
below). \cite{freeman1991parallel}

The intuition is also similar: RNCis the set of
all problems that have fast parallel randomized algorithms. \cite{freeman1991parallel}

Profiling:
\begin{itemize}
    \item gprof braucht exit() statt _exit und liefert keine ergebnisse
    \item valgrind sequenziert die einzelnen threads
    \item operf (oprofile) funktioniert nicht out of the box -> permission errors unter root, kann nicht ohne weiteres auf ebro gebaut oder installiert werden
    \item perf scheint zu funktionieren
    \item profiling ist teilweise hardwareabhängig und benötigt möglicherweise debug kernel
\end{itemize}

perf install and sources
https://askubuntu.com/questions/50145/how-to-install-perf-monitoring-tool#50148

Erkenntnisse:
\begin{itemize}
    \item dfs ist schwer zu parallelisieren
    \item bis in 80er eher als unparallelisierbar gehalten
    \item für lexikalische graphen nicht parallelisierbar
    \item parallele Algorithmen für bestimmte graphklassen gefunden (planare, disjoint)
    \item in Nicks Class (NC)
    \item auch algorithmus für allgemeine gerichtete graphen
    \item menge der knoten muss dabei allerdings for suche bekannt sein und kann nicht während der suche aufgedeckt werden
\end{itemize}

Eckpfeiler in der Implementierung
\begin{itemize}
    \item mutexe und Semaphore ersetzen -> keine signifikanten auswirkungen
    \item Überladungs fix in der parallelExploration -> pexplore wird benutzt
    \item benchmarking
    \begin{itemize}
        \item keine wesentlichen synchronizations bottlenecks im eigenen code
        \item keine wesentlichen idle threads
        \item threads entdecken ca gleich viele markierungen
        \item die meiste zeit wird im search and insert der threads verbracht
        \item benchmark in der unterklasse (hashingWrapperStore) sind inkonsistent
        \item möglicherweise arbeitet die Zeitmessung durch threading nicht so wie erwartet
        \item rückkehr zu profiling
    \end{itemize}
    \item profiling ergibt viel zeitverlust in libcalloc
    \item integration von mara
    \item macrobenchmark (zeitmessung für beispielnetz)
    \item singlethreadperformance von mara (compare and swap) ist sehr niedrig
    \item singlethreadperformance von mara (pthread) ist ähnlichähnlich niedrig
\end{itemize}

https://tex.stackexchange.com/questions/6834/change-paper-size-in-mid-document#6838

CHECKLIST ONE:

1. Is my thesis statement concise and clear?
2. Did I follow my outline? Did I miss anything?
3. Are my arguments presented in a logical sequence?
4. Are all sources properly cited to ensure that I am not plagiarizing?
5. Have I proved my thesis with strong supporting arguments?
6. Have I made my intentions and points clear in the essay?


CHECKLIST TWO:

1. Did I begin each paragraph with a proper topic sentence?
2. Have I supported my arguments with documented proof or examples?
3. Any run-on or unfinished sentences?
4. Any unnecessary or repetitious words?
5. Varying lengths of sentences?
6. Does one paragraph or idea flow smoothly into the next?
7. Any spelling or grammatical errors?
8. Quotes accurate in source, spelling, and punctuation?
9. Are all my citations accurate and in correct format?
10. Did I avoid using contractions? Use “cannot” instead of “can’t”, “do not” instead of “don’t”?
11. Did I use third person as much as possible? Avoid using phrases such as “I think”, “I guess”, “I suppose”
12. Have I made my points clear and interesting but remained objective?
13. Did I leave a sense of completion for my reader(s) at the end of the paper?

Fragen zur form
- Wie sollte groß und klein schreibung in den überschriften verwendet werden