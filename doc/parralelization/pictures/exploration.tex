\tikzset{%
    >={Latex[width=2mm,length=2mm]},
    % Specifications for style of nodes:
    base/.style = {
        rectangle, draw=black, 
        drop shadow,
        minimum width=2cm, 
        minimum height=1cm,
        text centered, 
        font=\sffamily
        },
    method/.style = {base, fill=red!30, minimum width=3.5cm},
    result/.style = {base, top color=green!20, bottom color=green!35},
    note/.style = {base, fill=yellow!40, align=left},
    class/.style = {base, rounded corners, minimum width=2.5cm, top color=blue!20, bottom color=blue!35, font=\ttfamily},
    arrowDescription/.style = {fill=white},
}
\begin{tikzpicture}[node distance=2cm, align=center]

    \node (Exploration)[class, thick, font=\bfseries]{\uline{Exploration}};
    \node (ExplorationJoin)[below of= Exploration]{};
    \node (NetStateJoin)[ left = of ExplorationJoin, xshift=0.5cm]{};
    \node (PropertyJoin)[ left = of NetStateJoin, xshift=-1cm]{};
    \node (StoreJoin)[ right = of ExplorationJoin, xshift=-0.5cm]{};
    \node (FireListJoin)[ right = of StoreJoin, xshift=1cm]{};
    \node (Property)[class, below =0.5cm of PropertyJoin]{Property};
    \node (NetState)[class, below =0.5cm of NetStateJoin]{NetState};
    \node (Store)[class, below =0.5cm of StoreJoin]{Store};
    \node (FireList)[class, below =0.5cm of FireListJoin]{FireList};
    \node (PlanningTask)[class, left = of Exploration]{PlanningTask};

    \node (methodprops)[method, above = of Exploration,  align=left]
        {DepthFirstSearch \\ FindPath \\ Sweepline \\ CoverBreadthFirst};
    \node (method)[method, anchor=south] at (methodprops.north){Method};    

    \node(result)[result, right= of Exploration]{True, False or Unknown\\ might not terminate};


    %%%% Notes %%%%
    \node(taskNotes)[note, above = of PlanningTask]
    { 
        \begin{varwidth}{3cm}
            A problem which has to be analyzed\\
            Something like:
        \begin{itemize}[noitemsep]
            \item Siphon/Trap
            \item Reachability
            \item Deadlock
            \item etc.
        \end{itemize}
        \end{varwidth}
    };
    \node(propertyNotes)[note, below = of Property]
    { 
        \begin{varwidth}{2.5cm}
        Can be true or false.\\
        Checked for a specific marking.
        \end{varwidth}
    };
    \node(NetStateNotes)[note, below = of NetState]
    { 
        \begin{varwidth}{2.5cm}
        The current marking.
        \end{varwidth}
    };
    \node(StoreNotes)[note, below = of Store]
    { 
        \begin{varwidth}{2.5cm}
            Keeps track of all discovered markings
        \end{varwidth}
    };
    \node(FireListNotes)[note, below = of FireList]
    { 
        \begin{varwidth}{2.5cm}
            All active transitions in the current marking
        \end{varwidth}
    };

    %%% Edges %%%
    \draw[->]	(Exploration) -- node[arrowDescription]{works on}(ExplorationJoin.center) -- (PropertyJoin.center) -- (Property);
    \draw[->]	(PlanningTask) -- node[arrowDescription]{calls} (Exploration);
    \draw[->]	(Exploration) -- node[arrowDescription]{uses} (methodprops);
    \draw[->]	(NetStateJoin.center) -- (NetState);
    \draw[->]	(StoreJoin.center) -- (Store);
    \draw[->]	(ExplorationJoin.center) -- (FireListJoin.center) -- (FireList);
    \draw[->]	(Exploration) -- node[arrowDescription]{returns}(result);
    
    \draw[-](PlanningTask) -- (taskNotes);
    \draw[-](Property) --(propertyNotes);
    \draw[-](NetState) --(NetStateNotes);
    \draw[-](Store) --(StoreNotes);
    \draw[-](FireList) --(FireListNotes);


\end{tikzpicture}