% !TEX root = ../main.tex
\section*{Parallele Zustandsraumsuche}
Mit dem LowLevelAnalyzer (LoLA) ist es durch Auswertung des Zustandsraums möglich, Eigenschaften von Petri-Netzen auszuwerten. Die dafür notwendige Suche wurde dabei bisher durch einen sequentiellen Algorithmus vorgenommen.

Im Hardwarebereich haben sich Prozessoren mit mehreren Kernen durchgesetzt. LoLAs sequentieller Ansatz kann deshalb oft das Rechenpotential moderner Prozessoren nicht ausschöpfen. Auch ein Versuch den vorhandenen Algorithmus zu parallelisieren brachte keine Performanceverbesserung.

In dieser Arbeit wird gezeigt das es möglich ist eine Performanceverbesserung durch Nutzung von mehreren Threads zu erzielen. Der vorherige Versuch wird analysiert und die Implementation so angepasst, dass eine Skalierung der Performance mit der Anzahl der genutzten Threads zu beobachten ist. Die Verbesserung wird an einem verbreiteten Petri-Netz gezeigt.

\section*{Parallel state space search}
The LowLevelAnalyzer (LoLA) is a useful tool to analyze properties of a Petri net by expanding and evaluating its state space. To search for new states LoLA makes use of a sequential depth-first search algorithm.

However, recent hardware development introduced multi-core processors that cannot be fully exploited with LoLAs implementation. For this reason, the previously single-threaded search algorithm was adapted to use multiple threads. Unfortunately, the sequential algorithm always beats the parallel in the matter of execution time for a yet unknown reason.

In this work, it is shown that it is possible to utilize multiple threads for the state space expansion. The previous parallel implementation of LoLA is analyzed and improved to achieve performance scaling with the number of threads used. The improvement is shown using a common Petri net.


\vfill

\begin{tabular}{ll}
	\bfseries Betreuer: & \parbox[t]{10cm}{\betreuer }\vspace{5mm} \\
	\bfseries Tag der Ausgabe: & 13.10.2017 \\
	\bfseries Tag der Abgabe: & 02.03.2018 \\
\end{tabular}
