\begin{figure}
  \centering
  \tikzset{
    place/.style={
      circle,
      thick,
      draw=blue!75,
      fill=blue!20,
      minimum size=6mm
    },
    transition/.style={
      rectangle,
      thick,
      draw=black,
      %fill=black,
      minimum width=8mm,
      minimum height=4.5mm,
      inner ysep=2pt
      }
    }            

  \begin{tikzpicture}[node distance=1.3cm,>=stealth',bend angle=45,auto]

      \begin{scope}[]
          \node [place,tokens=1] (w1) {};
          \node (center) [below of=w1]{};
          \node [place,tokens=0] (w2) [below of=center]{};

          \node [transition] (e1) [right of=center]{}
          edge [pre,bend right]    (w1)
          edge [post,bend left]  (w2);

          \node [transition] (e2) [left of=center]{}
          edge [pre,bend right]   (w2)
          edge [post,bend left]   (w1);

          \node(caption) [below of=e1,label=center:A]{};
      \end{scope}

      \begin{scope}[xshift=0.5cm, yshift=-4cm]
        \node [place,tokens=0] (w1) {};
        \node [transition] (e1) [left of=w1] {}
        edge [pre]                  (w1);
        \node(caption) [right of=w1,label=left:B]{};
      \end{scope}

      \begin{scope}[xshift=0.5cm, yshift=-5.5cm]
        \node [transition] (e1) {};
        \node [place,tokens=0] (w1) [left of=e1] {}
        edge [pre]                  (e1);
        \node(caption) [right of=e1,label=left:C]{};
      \end{scope}

      \begin{scope}[xshift=5cm]
        \node [place,tokens=2] (w1) {};
        \node [transition] (e1) [below of=w1]{}
        edge [pre]    (w1);
        \node [place,tokens=0] (w2-1) [below of=e1,xshift=-20]{}
        edge [pre]    (e1);
        \node [place,tokens=0] (w2-2) [below of=e1,xshift=20]{}
        edge [pre]    (e1);
        \node [transition] (e2-1) [below of=w2-1]{}
        edge [pre]    (w2-1);
        \node [transition] (e2-2) [below of=w2-2]{}
        edge [pre]    (w2-2);
        \node [place,tokens=0] (w3) [below of=e2-1,xshift=20]{}
        edge [pre]    (e2-1)
        edge [pre]    (e2-2);

        \node(caption) [right of=w3,label=left:D]{};
    \end{scope}

    \begin{scope}[xshift=8cm]
      \node [transition] (e1) [label=right:Transition]{};
      \node [place,tokens=0] (w1) [below of=e1,label=right:Place without tokens] {};
      \node [place,tokens=2] (w2) [below of=w1,label=right:Place with two tokens] {};
    \end{scope}

  \end{tikzpicture}
  \caption{
    Petri net examples\\
    \textbf{Net A}: Net can always fire one transition. One of the places will always have a token. If one has a token, the over will be unmarked.\\
    \textbf{Net B}: The transition cannot be fired because the place has no tokens. If the place would have n tokens, the transition could fire exactly n times. After that, the pictured state would be reached.\\
    \textbf{Net C}: Transition can always be fired since all preceding places (of which there are none) hold tokens. The place can have virtually infinite tokens.\\
    \textbf{Net D}: Fireing the top transition will consume one token from the top place and produce two tokens: one on the left place and one on the right place. At the final net state, the bottom place will hold four tokens and the others will hold none.
    }
    \label{petrNetExamples}
\end{figure}